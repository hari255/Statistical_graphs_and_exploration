% Options for packages loaded elsewhere
\PassOptionsToPackage{unicode}{hyperref}
\PassOptionsToPackage{hyphens}{url}
%
\documentclass[
]{article}
\usepackage{amsmath,amssymb}
\usepackage{lmodern}
\usepackage{iftex}
\ifPDFTeX
  \usepackage[T1]{fontenc}
  \usepackage[utf8]{inputenc}
  \usepackage{textcomp} % provide euro and other symbols
\else % if luatex or xetex
  \usepackage{unicode-math}
  \defaultfontfeatures{Scale=MatchLowercase}
  \defaultfontfeatures[\rmfamily]{Ligatures=TeX,Scale=1}
\fi
% Use upquote if available, for straight quotes in verbatim environments
\IfFileExists{upquote.sty}{\usepackage{upquote}}{}
\IfFileExists{microtype.sty}{% use microtype if available
  \usepackage[]{microtype}
  \UseMicrotypeSet[protrusion]{basicmath} % disable protrusion for tt fonts
}{}
\makeatletter
\@ifundefined{KOMAClassName}{% if non-KOMA class
  \IfFileExists{parskip.sty}{%
    \usepackage{parskip}
  }{% else
    \setlength{\parindent}{0pt}
    \setlength{\parskip}{6pt plus 2pt minus 1pt}}
}{% if KOMA class
  \KOMAoptions{parskip=half}}
\makeatother
\usepackage{xcolor}
\usepackage[margin=1in]{geometry}
\usepackage{color}
\usepackage{fancyvrb}
\newcommand{\VerbBar}{|}
\newcommand{\VERB}{\Verb[commandchars=\\\{\}]}
\DefineVerbatimEnvironment{Highlighting}{Verbatim}{commandchars=\\\{\}}
% Add ',fontsize=\small' for more characters per line
\usepackage{framed}
\definecolor{shadecolor}{RGB}{248,248,248}
\newenvironment{Shaded}{\begin{snugshade}}{\end{snugshade}}
\newcommand{\AlertTok}[1]{\textcolor[rgb]{0.94,0.16,0.16}{#1}}
\newcommand{\AnnotationTok}[1]{\textcolor[rgb]{0.56,0.35,0.01}{\textbf{\textit{#1}}}}
\newcommand{\AttributeTok}[1]{\textcolor[rgb]{0.77,0.63,0.00}{#1}}
\newcommand{\BaseNTok}[1]{\textcolor[rgb]{0.00,0.00,0.81}{#1}}
\newcommand{\BuiltInTok}[1]{#1}
\newcommand{\CharTok}[1]{\textcolor[rgb]{0.31,0.60,0.02}{#1}}
\newcommand{\CommentTok}[1]{\textcolor[rgb]{0.56,0.35,0.01}{\textit{#1}}}
\newcommand{\CommentVarTok}[1]{\textcolor[rgb]{0.56,0.35,0.01}{\textbf{\textit{#1}}}}
\newcommand{\ConstantTok}[1]{\textcolor[rgb]{0.00,0.00,0.00}{#1}}
\newcommand{\ControlFlowTok}[1]{\textcolor[rgb]{0.13,0.29,0.53}{\textbf{#1}}}
\newcommand{\DataTypeTok}[1]{\textcolor[rgb]{0.13,0.29,0.53}{#1}}
\newcommand{\DecValTok}[1]{\textcolor[rgb]{0.00,0.00,0.81}{#1}}
\newcommand{\DocumentationTok}[1]{\textcolor[rgb]{0.56,0.35,0.01}{\textbf{\textit{#1}}}}
\newcommand{\ErrorTok}[1]{\textcolor[rgb]{0.64,0.00,0.00}{\textbf{#1}}}
\newcommand{\ExtensionTok}[1]{#1}
\newcommand{\FloatTok}[1]{\textcolor[rgb]{0.00,0.00,0.81}{#1}}
\newcommand{\FunctionTok}[1]{\textcolor[rgb]{0.00,0.00,0.00}{#1}}
\newcommand{\ImportTok}[1]{#1}
\newcommand{\InformationTok}[1]{\textcolor[rgb]{0.56,0.35,0.01}{\textbf{\textit{#1}}}}
\newcommand{\KeywordTok}[1]{\textcolor[rgb]{0.13,0.29,0.53}{\textbf{#1}}}
\newcommand{\NormalTok}[1]{#1}
\newcommand{\OperatorTok}[1]{\textcolor[rgb]{0.81,0.36,0.00}{\textbf{#1}}}
\newcommand{\OtherTok}[1]{\textcolor[rgb]{0.56,0.35,0.01}{#1}}
\newcommand{\PreprocessorTok}[1]{\textcolor[rgb]{0.56,0.35,0.01}{\textit{#1}}}
\newcommand{\RegionMarkerTok}[1]{#1}
\newcommand{\SpecialCharTok}[1]{\textcolor[rgb]{0.00,0.00,0.00}{#1}}
\newcommand{\SpecialStringTok}[1]{\textcolor[rgb]{0.31,0.60,0.02}{#1}}
\newcommand{\StringTok}[1]{\textcolor[rgb]{0.31,0.60,0.02}{#1}}
\newcommand{\VariableTok}[1]{\textcolor[rgb]{0.00,0.00,0.00}{#1}}
\newcommand{\VerbatimStringTok}[1]{\textcolor[rgb]{0.31,0.60,0.02}{#1}}
\newcommand{\WarningTok}[1]{\textcolor[rgb]{0.56,0.35,0.01}{\textbf{\textit{#1}}}}
\usepackage{graphicx}
\makeatletter
\def\maxwidth{\ifdim\Gin@nat@width>\linewidth\linewidth\else\Gin@nat@width\fi}
\def\maxheight{\ifdim\Gin@nat@height>\textheight\textheight\else\Gin@nat@height\fi}
\makeatother
% Scale images if necessary, so that they will not overflow the page
% margins by default, and it is still possible to overwrite the defaults
% using explicit options in \includegraphics[width, height, ...]{}
\setkeys{Gin}{width=\maxwidth,height=\maxheight,keepaspectratio}
% Set default figure placement to htbp
\makeatletter
\def\fps@figure{htbp}
\makeatother
\setlength{\emergencystretch}{3em} % prevent overfull lines
\providecommand{\tightlist}{%
  \setlength{\itemsep}{0pt}\setlength{\parskip}{0pt}}
\setcounter{secnumdepth}{-\maxdimen} % remove section numbering
\ifLuaTeX
  \usepackage{selnolig}  % disable illegal ligatures
\fi
\IfFileExists{bookmark.sty}{\usepackage{bookmark}}{\usepackage{hyperref}}
\IfFileExists{xurl.sty}{\usepackage{xurl}}{} % add URL line breaks if available
\urlstyle{same} % disable monospaced font for URLs
\hypersetup{
  pdftitle={assignment\_5},
  pdfauthor={Harinath Reddy},
  hidelinks,
  pdfcreator={LaTeX via pandoc}}

\title{assignment\_5}
\author{Harinath Reddy}
\date{2022-11-16}

\begin{document}
\maketitle

\begin{Shaded}
\begin{Highlighting}[]
\FunctionTok{library}\NormalTok{(cdcfluview); }\FunctionTok{library}\NormalTok{(dplyr)}
\end{Highlighting}
\end{Shaded}

\begin{verbatim}
## 
## Attaching package: 'dplyr'
\end{verbatim}

\begin{verbatim}
## The following objects are masked from 'package:stats':
## 
##     filter, lag
\end{verbatim}

\begin{verbatim}
## The following objects are masked from 'package:base':
## 
##     intersect, setdiff, setequal, union
\end{verbatim}

\begin{Shaded}
\begin{Highlighting}[]
\FunctionTok{library}\NormalTok{(tsibble); }\FunctionTok{library}\NormalTok{(lubridate)}
\end{Highlighting}
\end{Shaded}

\begin{verbatim}
## 
## Attaching package: 'tsibble'
\end{verbatim}

\begin{verbatim}
## The following objects are masked from 'package:base':
## 
##     intersect, setdiff, union
\end{verbatim}

\begin{verbatim}
## Loading required package: timechange
\end{verbatim}

\begin{verbatim}
## 
## Attaching package: 'lubridate'
\end{verbatim}

\begin{verbatim}
## The following object is masked from 'package:tsibble':
## 
##     interval
\end{verbatim}

\begin{verbatim}
## The following objects are masked from 'package:base':
## 
##     date, intersect, setdiff, union
\end{verbatim}

\begin{Shaded}
\begin{Highlighting}[]
\FunctionTok{library}\NormalTok{(fable);}\FunctionTok{library}\NormalTok{(ggplot2)}
\end{Highlighting}
\end{Shaded}

\begin{verbatim}
## Loading required package: fabletools
\end{verbatim}

\begin{Shaded}
\begin{Highlighting}[]
\FunctionTok{library}\NormalTok{(feasts)}
\end{Highlighting}
\end{Shaded}

\begin{Shaded}
\begin{Highlighting}[]
\CommentTok{\# Prepare the data}
\NormalTok{usflu.raw }\OtherTok{\textless{}{-}} \FunctionTok{ilinet}\NormalTok{(}\StringTok{"national"}\NormalTok{, }\AttributeTok{years =} \DecValTok{2010}\SpecialCharTok{:}\DecValTok{2018}\NormalTok{)}
\FunctionTok{names}\NormalTok{(usflu.raw)}
\end{Highlighting}
\end{Shaded}

\begin{verbatim}
##  [1] "region_type"      "region"           "year"             "week"            
##  [5] "weighted_ili"     "unweighted_ili"   "age_0_4"          "age_25_49"       
##  [9] "age_25_64"        "age_5_24"         "age_50_64"        "age_65"          
## [13] "ilitotal"         "num_of_providers" "total_patients"   "week_start"
\end{verbatim}

\begin{Shaded}
\begin{Highlighting}[]
\NormalTok{usflu }\OtherTok{\textless{}{-}}\NormalTok{ usflu.raw }\SpecialCharTok{\%\textgreater{}\%}
  \FunctionTok{mutate}\NormalTok{(}
    \AttributeTok{date =} \FunctionTok{as.Date}\NormalTok{(}\FunctionTok{paste0}\NormalTok{(year, }\FunctionTok{sprintf}\NormalTok{(}\StringTok{"\%02d"}\NormalTok{, week), }\StringTok{"00"}\NormalTok{),}
\AttributeTok{format=}\StringTok{"\%Y\%W\%w"}\NormalTok{),}
\AttributeTok{dec\_date =} \FunctionTok{decimal\_date}\NormalTok{(week\_start),}
\AttributeTok{week =} \FunctionTok{yearweek}\NormalTok{(week\_start),}
\AttributeTok{time\_in\_year =}\NormalTok{ dec\_date}\SpecialCharTok{\%\%}\DecValTok{1}\NormalTok{)}\SpecialCharTok{\%\textgreater{}\%}
\NormalTok{  dplyr}\SpecialCharTok{::}\FunctionTok{filter}\NormalTok{(}\SpecialCharTok{!}\FunctionTok{is.na}\NormalTok{(dec\_date))}
\end{Highlighting}
\end{Shaded}

\begin{verbatim}
## Warning in strptime(x, format, tz = "GMT"): (0-based) yday 368 in year 2014 is
## invalid
\end{verbatim}

\begin{Shaded}
\begin{Highlighting}[]
\NormalTok{usflu.ts }\OtherTok{\textless{}{-}} \FunctionTok{as\_tsibble}\NormalTok{(usflu, }\AttributeTok{index =}\NormalTok{ week)}
\end{Highlighting}
\end{Shaded}

\begin{Shaded}
\begin{Highlighting}[]
\NormalTok{autoplot.weighted\_ili}\OtherTok{\textless{}{-}}\NormalTok{ usflu.ts }\SpecialCharTok{\%\textgreater{}\%} \FunctionTok{autoplot}\NormalTok{(weighted\_ili) }\SpecialCharTok{+} \FunctionTok{theme\_minimal}\NormalTok{() }\SpecialCharTok{+} 
  \FunctionTok{labs}\NormalTok{(}\AttributeTok{title=}\StringTok{"autoploy weight"}\NormalTok{,}
       \AttributeTok{x=}\StringTok{""}\NormalTok{, }\AttributeTok{y=}\StringTok{"weight "}\NormalTok{)}
\NormalTok{autoplot.weighted\_ili}
\end{Highlighting}
\end{Shaded}

\includegraphics{Assignment_5_files/figure-latex/unnamed-chunk-3-1.pdf}

\begin{Shaded}
\begin{Highlighting}[]
\NormalTok{gg\_season.weight }\OtherTok{\textless{}{-}}\NormalTok{ usflu.ts }\SpecialCharTok{\%\textgreater{}\%} \FunctionTok{gg\_season}\NormalTok{(weighted\_ili) }\SpecialCharTok{+} \FunctionTok{theme\_minimal}\NormalTok{() }\SpecialCharTok{+} 
  \FunctionTok{labs}\NormalTok{(}\AttributeTok{title=}\StringTok{"seasonal weight "}\NormalTok{, }\AttributeTok{x=}\StringTok{"season"}\NormalTok{, }\AttributeTok{y=}\StringTok{"weight "}\NormalTok{)}
\NormalTok{gg\_season.weight}
\end{Highlighting}
\end{Shaded}

\includegraphics{Assignment_5_files/figure-latex/unnamed-chunk-4-1.pdf}

\begin{Shaded}
\begin{Highlighting}[]
\NormalTok{acf.weight}\OtherTok{\textless{}{-}}\NormalTok{ usflu.ts }\SpecialCharTok{\%\textgreater{}\%} \FunctionTok{ACF}\NormalTok{(weighted\_ili) }\SpecialCharTok{\%\textgreater{}\%}
  \FunctionTok{autoplot}\NormalTok{() }\SpecialCharTok{+} \FunctionTok{theme\_minimal}\NormalTok{() }\SpecialCharTok{+} \FunctionTok{labs}\NormalTok{(}\AttributeTok{title=}\StringTok{"Daily new death ACF"}\NormalTok{,}
                         \AttributeTok{x=}\StringTok{"ACF"}\NormalTok{, }\AttributeTok{y=}\StringTok{"weight"}\NormalTok{)}
\NormalTok{acf.weight}
\end{Highlighting}
\end{Shaded}

\includegraphics{Assignment_5_files/figure-latex/unnamed-chunk-5-1.pdf}

\begin{Shaded}
\begin{Highlighting}[]
\NormalTok{pacf.weight}\OtherTok{\textless{}{-}}\NormalTok{ usflu.ts }\SpecialCharTok{\%\textgreater{}\%} \FunctionTok{PACF}\NormalTok{(weighted\_ili) }\SpecialCharTok{\%\textgreater{}\%}
  \FunctionTok{autoplot}\NormalTok{() }\SpecialCharTok{+} \FunctionTok{theme\_minimal}\NormalTok{() }\SpecialCharTok{+} \FunctionTok{labs}\NormalTok{(}\AttributeTok{title=}\StringTok{"Daily new death ACF"}\NormalTok{,}
                         \AttributeTok{x=}\StringTok{"PACF"}\NormalTok{, }\AttributeTok{y=}\StringTok{"weight"}\NormalTok{)}
\NormalTok{pacf.weight}
\end{Highlighting}
\end{Shaded}

\includegraphics{Assignment_5_files/figure-latex/unnamed-chunk-6-1.pdf}

** (a) AR(1) model with ϕ1 = 0.9 and σ2 = 1.

\begin{Shaded}
\begin{Highlighting}[]
\FunctionTok{tsibble}\NormalTok{(}\AttributeTok{idx =} \FunctionTok{seq\_len}\NormalTok{(}\DecValTok{200}\NormalTok{), }\AttributeTok{sim =} \DecValTok{1} \SpecialCharTok{+} \FunctionTok{arima.sim}\NormalTok{(}\FunctionTok{list}\NormalTok{(}\AttributeTok{ar =} \FunctionTok{c}\NormalTok{(}\FloatTok{0.9}\NormalTok{)), }\AttributeTok{n =} \DecValTok{200}\NormalTok{), }\AttributeTok{index =}\NormalTok{ idx) }\SpecialCharTok{\%\textgreater{}\%}
\FunctionTok{autoplot}\NormalTok{(sim) }\SpecialCharTok{+} \FunctionTok{ylab}\NormalTok{(}\StringTok{""}\NormalTok{) }\SpecialCharTok{+} \FunctionTok{ggtitle}\NormalTok{(}\StringTok{"AR(1)"}\NormalTok{)}
\end{Highlighting}
\end{Shaded}

\includegraphics{Assignment_5_files/figure-latex/unnamed-chunk-7-1.pdf}

** (a) AR(1) model with ϕ1 = 0.9 and σ2 = 1.

\begin{Shaded}
\begin{Highlighting}[]
\FunctionTok{tsibble}\NormalTok{(}\AttributeTok{idx =} \FunctionTok{seq\_len}\NormalTok{(}\DecValTok{200}\NormalTok{), }\AttributeTok{sim =} \DecValTok{1} \SpecialCharTok{+} \FunctionTok{arima.sim}\NormalTok{(}\FunctionTok{list}\NormalTok{(}\AttributeTok{ar =} \FunctionTok{c}\NormalTok{(}\FloatTok{0.8}\NormalTok{)), }\AttributeTok{n =} \DecValTok{200}\NormalTok{), }\AttributeTok{index =}\NormalTok{ idx) }\SpecialCharTok{\%\textgreater{}\%}
\FunctionTok{autoplot}\NormalTok{(sim) }\SpecialCharTok{+} \FunctionTok{ylab}\NormalTok{(}\StringTok{""}\NormalTok{) }\SpecialCharTok{+} \FunctionTok{ggtitle}\NormalTok{(}\StringTok{"MA(1)"}\NormalTok{)}
\end{Highlighting}
\end{Shaded}

\includegraphics{Assignment_5_files/figure-latex/unnamed-chunk-8-1.pdf}

\begin{Shaded}
\begin{Highlighting}[]
\FunctionTok{tsibble}\NormalTok{(}\AttributeTok{idx =} \FunctionTok{seq\_len}\NormalTok{(}\DecValTok{200}\NormalTok{), }\AttributeTok{sim =} \DecValTok{1} \SpecialCharTok{+} \FunctionTok{arima.sim}\NormalTok{(}\FunctionTok{list}\NormalTok{(}\AttributeTok{ar =} \FunctionTok{c}\NormalTok{(}\FloatTok{0.3}\NormalTok{,}\SpecialCharTok{{-}}\FloatTok{0.4}\NormalTok{)), }\AttributeTok{n =} \DecValTok{200}\NormalTok{), }\AttributeTok{index =}\NormalTok{ idx) }\SpecialCharTok{\%\textgreater{}\%}
\FunctionTok{autoplot}\NormalTok{(sim) }\SpecialCharTok{+} \FunctionTok{ylab}\NormalTok{(}\StringTok{""}\NormalTok{) }\SpecialCharTok{+} \FunctionTok{ggtitle}\NormalTok{(}\StringTok{"MA(2)"}\NormalTok{)}
\end{Highlighting}
\end{Shaded}

\includegraphics{Assignment_5_files/figure-latex/unnamed-chunk-9-1.pdf}

\begin{Shaded}
\begin{Highlighting}[]
\FunctionTok{library}\NormalTok{(IDDA)}
\FunctionTok{data}\NormalTok{(}\StringTok{"state.long"}\NormalTok{)}
\end{Highlighting}
\end{Shaded}

\begin{Shaded}
\begin{Highlighting}[]
\NormalTok{va\_state}\OtherTok{\textless{}{-}}\NormalTok{ IDDA}\SpecialCharTok{::}\NormalTok{state.long}\SpecialCharTok{\%\textgreater{}\%}
  \FunctionTok{filter}\NormalTok{(State}\SpecialCharTok{==}\StringTok{"Virginia"}\NormalTok{)}

\NormalTok{va\_state}
\end{Highlighting}
\end{Shaded}

\begin{verbatim}
## # A tibble: 345 x 7
##    State    Region Division           pop DATE       Infected Death
##    <chr>    <fct>  <fct>            <int> <date>        <int> <int>
##  1 Virginia South  South Atlantic 8517685 2020-12-31   349584  5032
##  2 Virginia South  South Atlantic 8517685 2020-12-30   344343  4982
##  3 Virginia South  South Atlantic 8517685 2020-12-29   340297  4918
##  4 Virginia South  South Atlantic 8517685 2020-12-28   336173  4857
##  5 Virginia South  South Atlantic 8517685 2020-12-27   333570  4850
##  6 Virginia South  South Atlantic 8517685 2020-12-26   329575  4833
##  7 Virginia South  South Atlantic 8517685 2020-12-25   327990  4816
##  8 Virginia South  South Atlantic 8517685 2020-12-24   323913  4788
##  9 Virginia South  South Atlantic 8517685 2020-12-23   319131  4757
## 10 Virginia South  South Atlantic 8517685 2020-12-22   314479  4701
## # ... with 335 more rows
\end{verbatim}

*** splitting the data into training and testing sets also seperating
data from dependent variable***

\begin{Shaded}
\begin{Highlighting}[]
\NormalTok{state.ts }\OtherTok{\textless{}{-}} \FunctionTok{as\_tsibble}\NormalTok{(state.long, }\AttributeTok{key =}\NormalTok{ State) }\SpecialCharTok{\%\textgreater{}\%}
\FunctionTok{group\_by}\NormalTok{(State) }\SpecialCharTok{\%\textgreater{}\%}
\FunctionTok{mutate}\NormalTok{(}\AttributeTok{Infected =}\NormalTok{ Infected}\SpecialCharTok{/}\DecValTok{1000}\NormalTok{) }\SpecialCharTok{\%\textgreater{}\%}
\FunctionTok{mutate}\NormalTok{(}\AttributeTok{YDA\_Infected =} \FunctionTok{lag}\NormalTok{(Infected, }\AttributeTok{order\_by =}\NormalTok{ DATE)) }\SpecialCharTok{\%\textgreater{}\%}
\FunctionTok{mutate}\NormalTok{(}\AttributeTok{YDA\_Death =} \FunctionTok{lag}\NormalTok{(Death, }\AttributeTok{order\_by =}\NormalTok{ DATE)) }\SpecialCharTok{\%\textgreater{}\%}
\FunctionTok{mutate}\NormalTok{(}\AttributeTok{Y.Infected =}\NormalTok{ Infected }\SpecialCharTok{{-}}\NormalTok{ YDA\_Infected) }\SpecialCharTok{\%\textgreater{}\%}
\FunctionTok{mutate}\NormalTok{(}\AttributeTok{Y.Death =}\NormalTok{ Death }\SpecialCharTok{{-}}\NormalTok{ YDA\_Death) }\SpecialCharTok{\%\textgreater{}\%}
\FunctionTok{mutate}\NormalTok{(}\AttributeTok{cum\_infected =} \FunctionTok{cumsum}\NormalTok{(Infected))}\SpecialCharTok{\%\textgreater{}\%}
\FunctionTok{mutate}\NormalTok{(}\AttributeTok{cum\_death =} \FunctionTok{cumsum}\NormalTok{(Death)) }\SpecialCharTok{\%\textgreater{}\%}
\NormalTok{dplyr}\SpecialCharTok{::}\FunctionTok{filter}\NormalTok{(}\SpecialCharTok{!}\FunctionTok{is.na}\NormalTok{(Y.Infected)) }\SpecialCharTok{\%\textgreater{}\%}
\NormalTok{dplyr}\SpecialCharTok{::}\FunctionTok{filter}\NormalTok{(}\SpecialCharTok{!}\FunctionTok{is.na}\NormalTok{(Y.Death)) }\SpecialCharTok{\%\textgreater{}\%}
\NormalTok{dplyr}\SpecialCharTok{::}\FunctionTok{select}\NormalTok{(}\SpecialCharTok{{-}}\FunctionTok{c}\NormalTok{(YDA\_Infected, YDA\_Death))}\SpecialCharTok{\%\textgreater{}\%}
\FunctionTok{filter}\NormalTok{(State}\SpecialCharTok{==}\StringTok{"Virginia"}\NormalTok{) }
\end{Highlighting}
\end{Shaded}

\begin{verbatim}
## Using `DATE` as index variable.
\end{verbatim}

\begin{Shaded}
\begin{Highlighting}[]
\NormalTok{Virginia.ts }\OtherTok{\textless{}{-}}\NormalTok{ state.ts }\SpecialCharTok{\%\textgreater{}\%}
\NormalTok{dplyr}\SpecialCharTok{::}\FunctionTok{filter}\NormalTok{(State }\SpecialCharTok{==} \StringTok{"Virginia"}\NormalTok{) }\SpecialCharTok{\%\textgreater{}\%}
\NormalTok{dplyr}\SpecialCharTok{::}\FunctionTok{select}\NormalTok{(Infected, Death, cum\_infected, cum\_death, Y.Death, Y.Infected)}
\end{Highlighting}
\end{Shaded}

\begin{verbatim}
## Adding missing grouping variables: `State`
\end{verbatim}

\begin{Shaded}
\begin{Highlighting}[]
\NormalTok{arima.fit }\OtherTok{\textless{}{-}}\NormalTok{Virginia.ts }\SpecialCharTok{\%\textgreater{}\%}
  \FunctionTok{model}\NormalTok{(}\AttributeTok{arima =} \FunctionTok{ARIMA}\NormalTok{(Y.Death }\SpecialCharTok{\textasciitilde{}} \FunctionTok{PDQ}\NormalTok{(}\DecValTok{0}\NormalTok{,}\DecValTok{0}\NormalTok{,}\DecValTok{0}\NormalTok{)))}
\FunctionTok{report}\NormalTok{(arima.fit)}
\end{Highlighting}
\end{Shaded}

\begin{verbatim}
## Series: Y.Death 
## Model: ARIMA(0,1,2) 
## 
## Coefficients:
##           ma1      ma2
##       -0.6529  -0.2045
## s.e.   0.0510   0.0500
## 
## sigma^2 estimated as 164.6:  log likelihood=-1361.58
## AIC=2729.15   AICc=2729.22   BIC=2740.67
\end{verbatim}

\hypertarget{selected-model-is-arima012-with-log-likelyhood-estimator-of--1361.58-which-is-infact-very-poor.}{%
\subsection{Selected Model is ARIMA(0,1,2) with Log-likelyhood estimator
of -1361.58 which is infact very
poor.}\label{selected-model-is-arima012-with-log-likelyhood-estimator-of--1361.58-which-is-infact-very-poor.}}

\begin{Shaded}
\begin{Highlighting}[]
\NormalTok{arima.fit }\SpecialCharTok{\%\textgreater{}\%} \FunctionTok{gg\_tsresiduals}\NormalTok{(}\AttributeTok{lag=}\DecValTok{36}\NormalTok{)}
\end{Highlighting}
\end{Shaded}

\includegraphics{Assignment_5_files/figure-latex/unnamed-chunk-15-1.pdf}

\begin{Shaded}
\begin{Highlighting}[]
\DocumentationTok{\#\# The resudials plot seems obvious that the error terms are normally distributed.}
\end{Highlighting}
\end{Shaded}


\end{document}
